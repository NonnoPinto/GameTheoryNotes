\section*{Cheat sheet}
This paper is a summary of Game Theory course of UniPD, a.y. 2023/24.\\It's not intended to substitute slides or book study.

\subsection*{Static games}
Supposing you know how to write the normal form of a game and find NE, here are some of the most important formulas:
\subsubsection*{Mixed strategies}
\begin{table}[!ht]
    \centering
    \begin{tabular}{ccccc}
        ~ & ~ & \multicolumn{3}{c}{Player B} \\
        ~ & ~ & ~ & \textit{q} & \textit{1-q} \\
        \multirow{3}{*}{\rotatebox{90}{Player A}} &
            ~ & ~ & T & N \\ \cline{4-5}
        ~ & \textit{p} & \multicolumn{1}{c|}{T} & \multicolumn{1}{c|}{3,3} & \multicolumn{1}{c|}{0,0} \\ \cline{4-5}
        ~ & \textit{1-p} & \multicolumn{1}{c|}{N} & \multicolumn{1}{c|}{0,0} & \multicolumn{1}{c|}{6,1} \\ \cline{4-5}
    \end{tabular}
\end{table}

If you have three players you can use the same method, but you need $n$ matrixes, one for each move of one player (write as "title" of every matrix the move of the player left out).

Small tip: if a player has always the same payoff in a matrix, you can write it as a single number, ignoring probabilities (since, once you collect the payoff, the sum of all probabilities is 1).

If it's a zero-sum game, you can just write payoff of one player, usually the first one, and the other player will have the opposite payoff.

\textbf{Find all additional Nash equilibria of this game in mixed strategies}
\mat{u_A(T,T)q + u_A(T,N)(1-q) = u_A(N,T)q + u_A(N,N)(1-q)}
\mat{u_B(T,T)p + u_B(N,T)(1-p) = u_B(T,N)p + u_B(N,N)(1-p)}
Find \m{p\text{ and }q}

If you have more than two players, probabilities would be $p_1, p_2, 1-p_1 - p_2$.

\subsection*{Multistage games}

\subsubsection*{Grim trigger strategy}
The Grim Trigger strategy goal is to punish the other player if he deviates from the chosen strategy. The goal is to find a $\delta$ s.t. the punishment is enough to avoid deviation.

Here the procedure for \textbf{infinitely repeated games}:
\begin{itemize}
    \item Chose the player who gains more from deviation
    \item Call $p^*$ the payoff of the chosen strategy
    \item Call $p$ the payoff of the other strategy (usually NE)
    \item Apply the formula below
    \item Solve for $\delta$
\end{itemize}
\mat{\frac{p^*}{1-\delta} \geq p + \frac{p \delta}{1-\delta}}

\subsubsection*{Stick and carrot}
Same as before, but for \textbf{finitely repeated games}:
\begin{itemize}
    \item Chose the player who gains more from deviation
    \item Let $p$ the payoff of the chosen strategy (usually cooperation, not necessary a NE)
    \item Let $p_s$ the payoff of deviating
    \item Let the payoff of the other strategy, usually stick and carrot, $p_s$ and $p_c$ respectively
    \item Apply the formula below
    \item Solve for $\delta$
\end{itemize}
The intuition is to calculate the payoff of following cooperation plus the reward, then compare it with the payoff of deviating and receiving the stick.
\mat{p + \delta p_c \geq p_d + \delta p_s}

\subsubsection*{Baeysian games}
For the extensive form of the game, you can use the same method as before, but you have to consider \textbf{Nature} as a player who makes the first move (typically, with probability \textit{p} and \textit{1-p}).

The tree will have, then, Nature as root, and all possibile strategies as chilren, repeted for each Nature's move. 

When it comes to write the normal form, you have to consider the probability of each strategy, which is the payoff of the stretegy times the probability of the path that leads to that strategy. For example, if one cell of the bi-matrix is AA, it means that player X will play A in any case: payoff is payoff of A in that leaf times probability of the path that leads to A.