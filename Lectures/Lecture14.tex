\section{Multistage games pt 2, stick-and-carrot strategies}

\Que{What is a strategy in a multistage game?}
\Ans[]{Each player has to specify what to do in the first stage and what to do in subsequent games depending on the outcome of previous games.}

\Que{Theorem for independent stages}
\Ans[]{Suppose \m{s^{(t)} = (s_1^{(1)}, \dots, s_n^{(t)})} is a NE strategy for stage $t$ of multistage game $\ml{G}$; then there is a SPE of $\ml{G}$ with equilibrium path \m{(s_1^{(1)}, \dots, s_n^{(t)})}. In other words, a strategy where each player always plays a NE, is a SPE.}

\Que{Theorem for linked stages 1}
\Ans[]{Every NE $s^*$ of multistage game \m{\ml{G} = (\ml{G}_1, \dots, \ml{G}_T)} requires that a NE is played in the last stage. In other words, in order for a strategy to be a NE, it has to have a NE in the last stage, since players already know all previous outcomes.}

\Que{Theorem for linked stages 2}
\Ans[]{If stage games \m{\ml{G}_1, \dots, \ml{G}_T} has all unique NE, then \m{\ml{G} = (\ml{G}_1, \dots, \ml{G}_T)} has unique SPE. In other words, if every stage has only one NE, this is the equilibrium path.

There is a counter intuitive consequence of these theorems: if the last stage has multiple NE, that enable non-NE strategies to be played in previous stages.}

\Que{What is the meaning of $\delta$?}
\Ans[$\delta$]{ is the measure of how much we care about the future. A larger $\delta$ means we care more about future payoffs.}

\Que{One-stage deviation principle}
\Ans[]{A one stage unimprovable strategy must be optimal. \textit{(proof in Lecture14)}}
\Spec{\textbf{Optimal strategy}: a strategy $s_i$ is optimal for player $i$ i $\fa$ information set $h_i$ there is no way to improve it (more formally: \m{\nexists s'_i / u_i(s'_i|\{h_i\} > u_i(s_i|\{h_i\}}).

\textbf{One-stage unimprovable strategy}: a strategy $s_i$ is one-stage unimprovable if there is no $s'_i$ that differes in one single stage s.t. \m{u_i(s'_i|\{h_i\} > u_i(s_i|\{h_i\}}}

\Que{Exercise 1}
\Ans[]{Consider multistage game $\ml{G}$ = ($\ml{G}_1$, $\ml{G}_2$), with $\ml{G}_1$ being the first stage, and $\ml{G}_2$ being the second (and last) stage:
\begin{itemize}
    \item Is it possible (for some $\ml{G}_1$ and $\ml{G}_2$) to find a SPE for $\ml{G}$ where a non-NE is played in G2?  \textbf{No}
    \item Is it possible (for some $\ml{G}_1$ and  $\ml{G}_2$) to find a NE for $\ml{G}$ where a non-NE is played in G2? \textbf{No}
    \item Is it possible (for some $\ml{G}_1$ and  $\ml{G}_2$) to find a SPE for $\ml{G}$ where a non-NE is played in G1? \textbf{Yes}
    \item Is it possible (for some $\ml{G}_1$ and  $\ml{G}_2$) to find a SPE for $\ml{G}$ where a strictly dominated strategy is played in G1? \textbf{Yes}
    \item What is the minimum number of NE in stage game $\ml{G}_2$ to enable a carrot-and-stick SPE in G? What characteristics should these NE have? \textbf{At least two NE (stick and carrot)}
\end{itemize}
\textit{Answers are mine, so feel free to mail me if they are wrong}
}

\Que{Exercise 2

Ashley and Brook live together. During the winter break they contemplate giving each other a nice gift (G) for Christmas or not (N). They know each other’s preferences so they are able to buy a gift for 10 euros that is worth like 100 euros for the other. They
make this decision independently and without telling each other. After Christmas, they also consider whether to celebrate New Year’s eve downtown (D) or stay home (H).
For the New Year’s eve celebration, they decide independently of each other in a coordination-game fashion. Staying home has utility of $0$ for both. Going downtown has utility of $50$. However, spending New Year’s eve apart from each other has utility of $-100$ for both. The total payoff of the players is the sum of the partial payoffs in each stage with a discount factor of $\delta$ for the second stage.
\begin{enumerate}
    \item Write down the normal form of both stages of the multi-stage game.
    \item Find a trivial subgame-perfect equilibrium of the game where the players just play a Nash equilibrium in all stages, without any strategic connection.
    \item Is there a strategically connected SPE of the whole game where Ashley and Brook give gifts to each other? If so, show the minimum required discount factor value $\delta$min for that to hold.
\end{enumerate}
}
\Ans[]{
\begin{enumerate}
    \item Normal form of stage game
        \begin{table}[!ht]
            \begin{tabular}{c|c|c}
              & G       & N       \\ \hline
            G & 90 90   & -10 100 \\ \hline
            N & 100 -10 & 0 0    
            \end{tabular}
            \quad
            \begin{tabular}{c|c|c}
              & H         & D       \\ \hline
            H & 0 0       & -100 -100\\ \hline
            D & -100 -100 & 50 50    
            \end{tabular}
        \end{table}
        Only NE in Stage 1 is $(N, N)$

        In stage two NE are: $(H, H)$ (stick) and $(D, D)$ (carrot)
    \item Trivial SPE is playing stages independently: $(NHHHH, NHHHH)$ (or with the carrot)
    \item Cooperative SPE: "play G at stage 1. If $(G, G)$ at stage 1, play D, otherwise play H". Sustainable if $u$(cooperative) + $\delta u$(carrot) $\geq u$(unilateral deviant) + $\delta u$(stick) \m{\longrightarrow 90 + 50\delta \geq 100 + 0\delta \longrightarrow \delta \geq \frac{1}{5}}
\end{enumerate}
}